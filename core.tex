\section{Pattern Matching Compilation}
\label{sec:core}

We take a fairly standard approach to compiling away pattern
matching. As we may match simultaneously against multiple
side-effecting computations, we must be somewhat careful about
order. Optionally, we can expose incomplete or ambiguous pattern
matching as concrete effects.

The target language of pattern matching compilation, Core Frank,
replaces multi-handlers with a combination of call-by-value functions,
case statements, and unary effect handlers.

The syntax of Core Frank is given in Figure~\ref{fig:core-syntax}.
%
\begin{figure}
Types
\begin{syntax}
\slab{values}       &U, V          &::=& D~\many{U} \mid \thunk{C} \mid X \\
\slab{computations} &C             &::=& R \mid V \to C \\
\slab{returners}    &R             &::=& \effbox{\sigs}V
\\[1ex]
\slab{quantifiers}  &Z             &::=& X \mid \varepsilon \\
\slab{polytypes}    &P             &::=& \forall \many{Z}.Q \\
\slab{thunks}       &Q             &::=& \thunk{C}
\\[1ex]
\slab{signatures}   &\sig~\many{X} &::=& \cdot \mid c : U(\many{V}), \sig~\many{X} \\
\slab{effects}      &\sigs         &::=&
  \emptyset \mid \sigs, \sig~\many{V} \mid \varepsilon
\\[1ex]
\slab{type environments}
                    &\Gamma        &::=& \cdot \mid \Gamma, x:V \mid f:P \\
\end{syntax}

Terms
\begin{syntax}
\slab{inferable values}&u       &::=& x \mid f \mid c \mid d \\
\slab{checkable values}&v       &::=& u \mid k~\many{v} \mid \thunk{e}
\\[1ex]
\slab{inferable computations}&d &::=& \force{u} \mid d~v \\
%                               \mid  \key{let}~x=d~\key{in}~d' \\
                             &&\mid& \key{letrec}~\many{f : P = e}~\key{in}~d \\
\slab{checkable computations} &e  &::= & v \mid \lambda x.e \\
   &&\mid& \key{case}~u~\key{of}~
             (k~\many{x_k} \mapsto e_k)_k \\
   &&\mid& \key{handle}~d~\key{with}~
             \bstack
             (\handle{c~\many{x_c}}{g_c} \mapsto e_c)_c \\
             \medvert~ x        \mapsto e \\
             \estack \\
\end{syntax}
\caption{Core Frank Syntax}
\label{fig:core-syntax}
\end{figure}
%
The Core Frank typing rules are given in Figure~\ref{fig:core-typing}.
%
\begin{figure*}[float]
$\boxed{\effs{C}{\sigs}}$

\begin{mathpar}
\inferrule
  { }
  {\effs{\effbox{\sigs}V}{\sigs}}

\inferrule
  {\effs{C}{\sigs}}
  {\effs{V \to C}{\sigs}}
\end{mathpar}

$\boxed{\makes{\Gamma}{\sigs}{u}{V}}$
\begin{mathpar}
\inferrule
  {x:V \in \Gamma}
  {\makesgs{x}{V}}

%% \inferrule
%%   {\dom(\theta) = \set{\varepsilon} \cup \FV(V) \backslash (\FV(\Gamma) \cup \FV(\sigs)) \\
%%    \theta(\varepsilon) = \sigs}
%%   {\makes{\Gamma, x:V}{\sigs}{x}{\theta(V)}}

\inferrule
  {f:\forall \many{Z}.Q \in \Gamma \\
   \dom(\theta) = \set{\many{Z}} \\
   \varepsilon \in \dom(\theta) \implies \theta(\varepsilon) = \sigs}
  {\makesgs{f}{\theta(Q)}}

\inferrule
  {c : U(\many{V}) \in \sigs}
  {\makesgs{c}{\thunk{\many{V} \to \effbox{\sigs}U}}}

\inferrule
  {\cangs{d}{\effbox{\sigs}V}}
  {\makesgs{d}{V}}
\end{mathpar}

$\boxed{\has{\Gamma}{\sigs}{V}{v}}$

\begin{mathpar}
\inferrule
  {\makesgs{u}{U} \\ U = V}
  {\hasgs{V}{u}}

\inferrule
  {(\hasgs{V_i}{v_i})_i \\
   k~\many{V} \in D~\many{U}}
  {\hasgs{D~\many{U}}{k~\many{v}}}

\inferrule
  {\does{\Gamma}{C}{e}}
  {\hasgs{\thunk{C}}{\thunk{e}}}
\end{mathpar}

$\boxed{\can{\Gamma}{\sigs}{d}{C}}$

\begin{mathpar}
\inferrule
  {\makesgs{u}{\thunk{C}} \\ \effs{C}{\sigs}}
  {\cangs{\force{u}}{C}}

\inferrule
  {\cangs{d}{V \to C} \\
   \has{\Gamma}{\sigs}{V}{v}}
  {\can{\Gamma}{\sigs}{d~v}{C}}

%% \inferrule
%%   {c : U(\many{V}) \in \sigs \\
%%    (\hasgs{V_i}{v_i})_i}
%%   {\cangs{c~\many{v}}{\effbox{\sigs}U}}

%% \inferrule
%%   {\cangs{d}{V} \\
%%    \can{\Gamma, x:V}{\sigs}{d'}{C}}
%%   {\cangs{\key{let}~x=d~\key{in}~d'}{C}}

\inferrule
  {(\does{\Gamma, \many{f : \forall \many{Z}.Q}}{e_i}{Q_i})_i \\
   \can{\Gamma, \many{f : \forall \many{Z}.Q}}{\sigs}{d}{C}}
  {\cangs{\key{letrec}~\many{f : \forall \many{Z}.Q = e}~\key{in}~d}{C}}
\end{mathpar}

$\boxed{\doesg{C}{e}}$

\begin{mathpar}
\inferrule
  {\hasgs{V}{v}}
  {\doesg{\effbox{\sigs}V}{v}}

\inferrule
  {\does{\Gamma, x:V}{C}{e}}
  {\doesg{V \to C}{\lambda x.e}}

\inferrule
  {\makesgs{u}{D~\many{U}} \\
   \\ \effs{C}{\sigs} \\
   (\does{\Gamma, \many{x_k}:\many{V}}{C}{e_k})_{k~\many{V} \in D~\many{U}}}
  {\doesg{C}{\key{case}~ u ~\key{of}~
               (k~\many{x_k} \mapsto e_k)_k}}

\inferrule
  {\sigs' \text{ closed} \\
   \effs{C}{\sigs} \\
   \can{\Gamma}{\sigs \oplus \sigs'}{d}{\effbox{\sigs \oplus \sigs'}{V}} \\
   (\does{\Gamma, \many{x_c}:\many{V}, g_c:\thunk{U \to \effbox{\sigs \oplus \sigs'}{V}}}
         {C}{e_c})_{c : U(\many{V}) \in \sigs'} \\
   \does{\Gamma, x:V}{C}{e}}
  {\does{\Gamma}
         {C}{\key{handle}~ d ~\key{with}~
               (\handle{c~\many{x_c}}{g_c} \mapsto e_c)_c \medvert
                x \mapsto e}}
\end{mathpar}

\caption{Core Frank Typing Rules}
\label{fig:core-typing}
\end{figure*}


%% The syntax of Core Frank is given in Figure~\ref{fig:core-syntax}.

%% \begin{figure}
%% Types
%% \begin{syntax}
%% \slab{values}       &U, V   &::=& D~\many{U} \mid  \effbox{\sigs}\thunk{C} \mid X \\
%% \slab{computations} &C      &::=& \rt{V} \mid V \to C
%% \\[1ex]
%% \slab{polytypes}    &P      &::=& \forall \varepsilon \many{X}.Q \\
%% \slab{thunks}       &Q      &::=& \effbox{\sigs}\thunk{C}
%% \\[1ex]
%% \slab{signatures}   &\sig~\many{X} &::=& \cdot \mid c : U(\many{V}), \sig~\many{X} \\
%% \slab{effects}      &\sigs  &::=&
%%   \emptyset \mid \sigs, \sig~\many{V} \mid \varepsilon
%% \\[1ex]
%% \slab{type environments}     &\Gamma &::=& \cdot \mid \Gamma, x:V \mid f:P \\
%% \end{syntax}

%% Terms
%% \begin{syntax}
%% \slab{inferable values}       &u  &::= & x \mid f \mid d                           \\
%% \slab{checkable values}       &v  &::= & u \mid k~\many{v} \mid \thunk{e}
%% \\[1ex]
%% \slab{inferable computations} &d  &::= & \force{u} \mid d~v \mid c~\many{v}
%%                                    \mid  \key{let}~x=d~\key{in}~d' \\
%%                               &   &\mid& \key{letrec}~\many{f : P = e}~\key{in}~d \\
%% \slab{checkable computations} &e  &::= & v \mid \lambda x.e \\
%%   &&\mid& \key{case}~u~\key{of}~
%%             (k~\many{x_k} \mapsto e_k)_k \\
%%   &&\mid& \key{handle} ~d~ \key{with}
%%             \bstack
%%             ~~(\handle{c~\many{x_c}}{g_c} \mapsto e_c)_c \\
%%             \medvert~ x        \mapsto e \\
%%             \estack \\
%% \end{syntax}
%% \caption{Core Frank Syntax}
%% \label{fig:core-syntax}
%% \end{figure}


%% A superficial change to the types is that rather than annotating value
%% returning computations with effects, we shift such labels to the thunk
%% containing the computation. The former design seems more convenient to
%% program with, which is why we adopt it in the source language. The
%% latter design leads to a slightly more uniform presentation of the
%% type rules (the checking judgement for computations now has the same
%% shape as the checking judgement for values, and we no longer need the
%% $\judgeword{does}$ judgement).
%
%% The typing rules of Core Frank are given in
%% Figure~\ref{fig:core-syntax}.

%%%%% Core Frank

%% \begin{figure*}[float]
%% $\boxed{\makes{\Gamma}{\sigs}{u}{V}}$
%% \begin{mathpar}
%% \inferrule
%%   {x:V \in \Gamma}
%%   {\makesgs{x}{V}}

%% %% \inferrule
%% %%   {\dom(\theta) = \set{\varepsilon} \cup \FV(V) \backslash (\FV(\Gamma) \cup \FV(\sigs)) \\
%% %%    \theta(\varepsilon) = \sigs}
%% %%   {\makes{\Gamma, x:V}{\sigs}{x}{\theta(V)}}

%% \inferrule
%%   {f:\forall \varepsilon \many{X}.V \in \Gamma \\
%%    \dom(\theta) = \set{\varepsilon, \many{X}} \\
%%    \theta(\varepsilon) = \sigs}
%%   {\makesgs{f}{\theta(V)}}

%% \inferrule
%%   {\cangs{d}{\rt{V}}}
%%   {\makesgs{d}{V}}
%% \end{mathpar}

%% $\boxed{\has{\Gamma}{\sigs}{V}{v}}$

%% \begin{mathpar}
%% \inferrule
%%   {\makesgs{u}{V}}
%%   {\hasgs{V}{u}}

%% \inferrule
%%   {(\hasgs{V_i}{v_i})_i \\
%%    k~\many{V} \in D~\many{U}}
%%   {\hasgs{D~\many{U}}{k~\many{v}}}

%% \inferrule
%%   {\cdoes{\Gamma}{\sigs'}{C}{e}}
%%   {\hasgs{\effbox{\sigs'}\thunk{C}}{\thunk{e}}}
%% \end{mathpar}

%% $\boxed{\can{\Gamma}{\sigs}{d}{C}}$

%% \begin{mathpar}
%% \inferrule
%%   {\makesgs{u}{\effbox{\sigs}\thunk{C}}}
%%   {\cangs{\force{u}}{C}}

%% \inferrule
%%   {\cangs{d}{V \to C} \\
%%    \has{\Gamma}{\sigs}{V}{v}}
%%   {\cangs{d~v}{C}}

%% \inferrule
%%   {c : U(\many{V}) \in \sigs \\
%%    (\hasgs{V_i}{v_i})_i}
%%   {\cangs{c~\many{v}}{\rt{U}}}

%% \inferrule
%%   {\cangs{d}{\effbox{\sigs}V} \\
%%    \can{\Gamma, x:V}{\sigs}{d'}{C}}
%%   {\cangs{\key{let}~x=d~\key{in}~d'}{C}}

%% \inferrule
%%   {(\does{\Gamma, \many{f : \forall \varepsilon \many{X}.Q}}{e_i}{Q_i})_i \\
%%    \can{\Gamma, \many{f : \forall \varepsilon \many{X}.Q}}{\sigs}{d}{C}}
%%   {\cangs{\key{letrec}~\many{f : \forall \varepsilon \many{X}.Q = e}~\key{in}~d}{C}}
%% \end{mathpar}

%% $\boxed{\cdoesgs{C}{e}}$

%% \begin{mathpar}
%% \inferrule
%%   {\hasgs{V}{v}}
%%   {\cdoesgs{\rt{V}}{v}}

%% \inferrule
%%   {\cdoes{\Gamma, x:V}{\sigs}{C}{e}}
%%   {\cdoesgs{V \to C}{\lambda x.e}}

%% \inferrule
%%   {\makesgs{u}{D~\many{U}} \\
%%    (\cdoes{\Gamma, \many{x_k}:\many{V}}{\sigs}{C}{e_k})_{k~\many{V} \in D~\many{U}}}
%%   {\cdoesgs{C}{\key{case}~ u ~\key{of}~
%%                (k~\many{x_k} \mapsto e_k)_k}}

%% \inferrule
%%   {\sigs' \text{ closed} \\
%%    \can{\Gamma}{\sigs \oplus \sigs'}{d}{\rt{V}} \\
%%    (\cdoes{\Gamma, \many{x_c}:\many{V}, g_c:\effbox{\sigs \oplus \sigs'}\thunk{U \to \rt{V}}}
%%          {\sigs}{C}{e_c})_{c : U(\many{V}) \in \sigs'} \\
%%    \cdoes{\Gamma, x:V}{\sigs}{C}{e}}
%%   {\cdoes{\Gamma}{\sigs}
%%          {C}{\key{handle}~ d ~\key{with}~
%%                (\handle{c~\many{x_c}}{g_c} \mapsto e_c)_c \medvert
%%                 x \mapsto e}}
%% \end{mathpar}

%% \caption{Core Frank Typing Rules}
%% \label{fig:core-typing}
%% \end{figure*}

Multi-handlers in Frank become curried functions over suspended
computations in Core Frank.
%
Shallow pattern matching on a single request becomes unary effect
handling. Shallow pattern matching on a datatype value becomes case
analysis. Nested pattern matching on multiple computations is realised
as a pattern matching tree constructed from handlers and case
statements. 

Standard algorithms for pattern matching compilation
apply~(e.g. \cite{Augustsson85} or \cite{Maranget08}). Rather than
committing to a particular one, we outline how a pattern matching
compiler fits into our setting, what input it takes, and what kind of
output it must produce.

\begin{sloppypar}
Given a Frank expression $\thunk{e}$ such that $\doesg{R_1 \dots R_n
  \to R}{e}$
%
we compile it to an equivalent Core Frank expression $\thunk{\pc{e}}$.
%
First we expand all of the clauses in $e$ to yield an $n$ column
pattern matrix. For instance, suppose the arguments have types
$\effbox{\var{Send~Char}, \var{Abort}}\var{Unit}$ and
$\effbox{\var{Receive~Char}}$, and we have the following clauses:
\end{sloppypar}
\[
\ba{@{}l@{~}l@{~}l@{}}
  (\handle{\var{send}~x}{s}) & (\handle{\var{receive}}{r}) & \mapsto e_1 \\
  (\handle{\var{send}~x}{s}) & z & \mapsto e_2 \\
  (\handle{\var{abort}}{s})  & e_3 \\
  \var{unit} & e_4 \\
\ea
\]
%
then this becomes:
%
\[
\ba{@{}l@{\quad}l@{~\mapsto~}l@{}}
  (\handle{\var{send}~x}{s}) & (\handle{\var{receive}}{r}) & e_1 \\
  (\handle{\var{send}~x}{s}) & z                           & e_2 \\
  (\handle{\var{abort}}{s})  & \force{y} & \var{bind}~\force{y}~e_3 \\
  \var{unit}                 & \force{y} & \var{bind}~\force{y}~e_4 \\
\ea
\]
%
where $\var{bind}$ is defined in Section~\ref{sec:examples}. The
missing patterns have been inserted as thunk patterns, which match any
computation. Invoking $\var{bind}$ allows us to forward the
computation bound by the thunk pattern to the existing continuation.
%
Next we generate a vector of fresh variables, one for each argument.
\[
\ba{@{}l@{~}l@{}}
  x_0 & x_1
\ea
\]
The goal of pattern matching compilation is to generate a pattern
matching tree that matches the variable vector against all of the
patterns in the pattern matrix in the correct order.

In Frank, pattern matching trees $M$ are built up from leaves, case
analysis, and handlers.
%% \begin{equations}
%% M &::= & \many{e} \\
%%   &\mid& \key{case}~(x:D~\many{U})~\key{of}~(k~\many{x_k} \mapsto M_k)_{k \in D} \\
%%   &\mid& \key{handle}~(\force{x}:\effbox{\sigs}V)~\key{with}~
%%           (\handle{c~\many{x_c}}{g_c} \mapsto M_c)_{c \in \sigs} \medvert
%%           x        \mapsto M
%% \end{equations}%
\begin{equations}
M &::= & \many{e} \\
  &\mid& \key{case}~x~\key{of}~(k~\many{x_k} \mapsto M_k)_{k \in D} \\
  &\mid& \key{handle}~\force{x}~\key{with}~
          (\handle{c~\many{x_c}}{g_c} \mapsto M_c)_{c \in \sigs} \medvert
          x        \mapsto M
\end{equations}%
The leaves consist of a sequence of checkable computation
expressions. Each element corresponds to one way of matching all of
the patterns. If there exists a leaf with no elements, then the
pattern matching is incomplete; if there exists a leaf with multiple
elements, then the pattern matching is ambiguous.
%
Our default strategy (as indicated by the thunk introduction rule) is
to class incomplete pattern matching as a type error, and to keep only
the first element in the case of ambiguous pattern matching.
%
Our example generates the following pattern matching tree:
\[
M = \bstack
    \key{handle}~\force{x_0}~\key{with} \\
    \quad \handle{\var{send}~x}{s} \mapsto \\
    \quad\quad \key{handle}~\force{x_1\!}~\key{with} \\
    \quad\quad\quad \begin{clauses}
                    \handle{\var{receive}}{r} &\mapsto& e_1 \\
                    z                         &\mapsto& e_2 \\
                    \end{clauses} \\
    \quad \handle{abort}{s} \mapsto \\
    \quad\quad \key{handle}~\force{x_1\!}~\key{with} \\
    \quad\quad\quad \begin{clauses}
                    \handle{receive}{r} &\mapsto&
                      \var{bind}~(\force{r}~\force{receive})~e_3 \\
                    z &\mapsto& \var{bind}~z~e_3 \\
                    \end{clauses} \\
    \quad  y \mapsto \\
    \quad\quad  \key{case}~y~\key{of} \\
    \quad\quad\quad  \var{unit} \mapsto \key{handle}~\force{x_1\!}~\key{with} \\
    \quad\quad\quad\quad \begin{clauses}
                         \handle{receive}{r} &\mapsto&
                            \var{bind}~(\force{r}~\force{receive})~e_4 \\
                         z &\mapsto& \var{bind}~z~e_4 \\
                         \end{clauses} \\
\estack
\]
Each thunk pattern has been expanded out to explicitly list all of the
cases according to its type. We obtain the corresponding Core Frank
code by abstracting over the fresh variables.
%
\[
\lambda x_0~x_1.M
\]
Some pattern matching operations reorder columns as an
optimisation. Column reordering is not in general a valid optimisation
in Frank. This is because commands in the ambient effects, but not in
the argument effects, are implicitly forwarded, and the order in which
they are forwarded is left-to-right. (The forwarding behaviour is made
precise in the Section~\ref{sec:semantics}.)

Of course, because Core Frank takes values as arguments whereas Frank
takes computations, each argument must be wrapped in a thunk
constructor.
%
The type translation is given simply by the homomorphic extension of the
following equation on function types:
\begin{equations}
\pc{\effbox{\sigs}V} \to \pc{C} &=& \thunk{\effbox{\pc{\sigs}}\pc{V}} \to \pc{C} \\
\end{equations}

A correct pattern matching translation $\pc{-}$ from Frank to Core
Frank should be type preserving.
\begin{itemize}
\item If $\makesgs{u}{V}$ then $\makes{\pc{\Gamma}}{\pc{\sigs}}{\pc{u}}{\pc{V}}$.
\item If $\hasgs{V}{v}$   then $\has{\pc{\Gamma}}{\pc{\sigs}}{\pc{V}}{\pc{v}}$.
\item If $\cangs{d}{C}$   then $\can{\pc{\Gamma}}{\pc{\sigs}}{\pc{d}}{\pc{C}}$.
%%\item If $\doesg{C}{e}$  then $\does{\pc{\Gamma}}{\pc{C}}{\pc{e}}$.
\end{itemize}

\subsection{Incomplete and Ambiguous Pattern Matching as Effects}

As an extension to Frank, we might allow incomplete and ambiguous
pattern matching. The former may be permitted if the ambient effects
contain the $\var{Abort}$ signature, in which case incomplete patterns
are translated into the $\var{abort}:\var{Zero}$ command, which can
then be handled however the programmer wishes. Similarly, we can define
a $\var{choice}:X(X, X)$ command, in order to allow ambiguous pattern
matches to be handled by the programmer.

%% \begin{equations}
%% \pc{D~\many{U}) &=& D~\pc{\many{U}} \\
%% \pc{\thunk{C}}  &=& \thunk{\pc{C}} \\
%% \pc{X} &=& X
%% \\[1ex]
%% \pc{\effbox{\sigs}V} &=& \effbox{\pc{\sigs}}\pc{V} \\
%% \pc{\effbox{\sigs}V} \to \pc{C} &=& \thunk{\effbox{\pc{\sigs}}\pc{V}} \to \pc{C} \\
%% \\[1ex]
%% \pc{\emptyset} &=& \emptyset \\
%% \pc{\sigs, S~\many{V}} &=& \pc{\sigs}, S~\pc{\many{V} \\
%% \pc{\varepsilon} &=& \varepsilon \\
%% \\[1ex]
%% \pc{\cdot} &=& \cdot \\
%% \pc{c:U(\many{V}), S~\many{X}} &=& 
%% \end{equations}

%%  LocalWords:  Lindley Conor Strathclyde Plotkin Pretnar's Multi et
%%  LocalWords:  effectful modularity polymorphism Kammar al Pretnar
%%  LocalWords:  Hindley Milner Oury SML Haskell monadic Kiselyov DSL
%%  LocalWords:  Idris Brady's equational multi unary al's datatypes
%%  LocalWords:  datatype Vn typecheck Ri catter polytypes inferable
%%  LocalWords:  checkable letrec monomorphic reifying desugars ret
%%  LocalWords:  homomorphic monothunks reified pre rewrapped monads
%%  LocalWords:  denotationally Swierstra Visscher Filinski Daan Koka
%%  LocalWords:  Filinski's Leijen's Swamy monad
