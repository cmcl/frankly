\section{Denotational Semantics for Frank}
\label{sec:denote}

The implementation has a denotational semantics described by an untyped
evaluator which performs the handling of the effects by monadicly threading
computations through the thing

we have a compiler that is structured like parser then typechecker
thenddddddddd untyped evaluator the untyped evaluator does things on a
mid-level language that has be desugared somewhat

I have some vague notion of what i want here a monadic presentation of the
denotational semantics everything is a value the operations push unhandled
computations up the tree and a final value is returned make a point of
mentioning that this is with the full frank type system and so works for
multi-handlers not just unary handlers

some questions: how does this compare to ``traditional'' denotational
semantics? i.e. is it usually monadic? Is the monadic aspect even that
original/novel/noteworthy?

a figure describing for each construct its interpretation function with
suitable notation

denotation of contexts/environments
denotation of terms: values and computations
anything else?

%%  LocalWords:  Lindley Conor Strathclyde Plotkin Pretnar's Multi et
%%  LocalWords:  effectful modularity polymorphism Kammar al Pretnar
%%  LocalWords:  Hindley Milner Oury SML Haskell monadic Kiselyov DSL
%%  LocalWords:  Idris Brady's equational multi unary al's datatypes
%%  LocalWords:  datatype Vn typecheck Ri catter polytypes inferable
%%  LocalWords:  checkable letrec monomorphic reifying desugars ret
%%  LocalWords:  homomorphic monothunks reified pre rewrapped monads
%%  LocalWords:  denotationally Swierstra Visscher Filinski Daan Koka
%%  LocalWords:  Filinski's Leijen's Swamy monad
