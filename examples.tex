\section{Introducing Frank}
\label{sec:examples}

Frank is a functional programming language with effects and handlers
in the style of Eff controlled by a type system inspired by Levy's
call-by-push-value~\cite{Levy2004}.
%
Doing and Being are clearly separated, and managed by distinguished
notions of computation and value types.

Concrete values live in inductive datatypes.
%
\begin{verbatim}
data List X
   = nil
   | cons X (List X)

data Zero =

data Unit = unit
\end{verbatim}

We can write perfectly ordinary functional programs, with (compulsory)
type signatures.
%
\begin{verbatim}
append : List X -> List X -> List X
append nil       ys = ys
append (x :: xs) ys = cons x (append xs ys)
\end{verbatim}

Higher-order functions are passed suspended computations. Braces are
"suspenders".
%
\begin{verbatim}
map : {X -> Y} -> List X -> List Y
map f nil         = nil
map f (cons x xs) = cons (f x) (map f xs)
\end{verbatim}

A value type \verb!V! is a datatype \verb!D V1 ... Vn!, a suspended
computation type \verb!{C}!, or a type variable \verb!X!.
%
A computation type can be a function type \verb!V -> C! or an
effect-annotated value type \verb![S1 ... Sn]V!, where an empty
bracket may be omitted.

Effects are collections of signatures, which describe a choice of
commands. Here are some simple signatures.
%
\begin{verbatim}
sig Send X
  = send : Unit(X)

sig Receive X
  = receive : X

sig Abort
  = aborting : Zero
\end{verbatim}
%
The \verb!send! command takes an argument of type \verb!X! and returns
a value of type \verb!Unit!. The \verb!receive! command returns a
value of type \verb!X!. The abort command returns an element of the
empty type, \verb!Zero!.

Frank is a call-by-value language, but it naturally distinguishes
\verb![]V!, the type of suspended pure computations which deliver a
\verb!V!, from \verb!V! itself. We can thus define a kind of
"semicolon" just as the function which ignores its first argument.
%
\begin{verbatim}
semi : X Y -> Y
semi x y = y
\end{verbatim}

Frank has effect polymorphism, enough to allow higher-order functions
to pass effect permissions to their parameters. The following uses map
to send a list of things, one at a time.
%
\begin{verbatim}
sends : List X -> [Send X]Unit
sends xs = semi (map send xs) unit
\end{verbatim}
%
The reason this type checks at all is because \verb!map! is implicitly
polymorphic in its effects.

The following does not typecheck, because the \verb!Send! effect is
not permitted in the return type of \verb!bad!.

\begin{verbatim}
bad : List X -> Unit
bad xs = semi (map send xs) unit
\end{verbatim}

Writing control operators is not too tricky.
\begin{verbatim}
bind : X -> {X -> Y} -> Y
bind x f = f x
\end{verbatim}

We can use \verb!bind! to define a polymorphic \verb!abort! function.
\begin{verbatim}
abort : [Abort]X
abort = bind aborting! {}
\end{verbatim}
The term \verb!{}! denotes a suspended computation containing an empty
collection of pattern matching clauses covering the \verb!Zero! return
type of \verb!aborting!.

Here is a computation which receives and concatenates lists until one
is empty.
%
\begin{verbatim}
catter : [Receive (List X)]List X
catter = bind receive!
           { nil -> nil
           | xs  -> append xs catter!
           }
\end{verbatim}
%
The command \verb!receive! is a suspended computation of type
\verb!{[Receive (List X)]List X}! that delivers a list when
forced. The notation \verb|receive!| forces a thunk.

If \verb!f! is a suspended computation of function type, then
\verb!f x! is syntactic sugar for \verb|f! x|. Thus we need only
explicitly force computations that take no arguments.

Effects are handled by special functions called \emph{effect
  handlers}. Effect handlers in Frank can take multiple computations
as arguments, hence they are \emph{multi-handlers}. In fact standard
functions like the ones we have seen so far are just special cases of
multi-handlers in which the all the arguments are pure.

A multi-handler has type \verb!R1 -> ... -> Rm -> R! where each
\verb!Ri! and \verb!R! is an effect-annotated value type. For
instance, we can write a \verb!pipe! multi-handler which handles
\verb!send! commands from one computation by matching them against
corresponding \verb!receive! commands from another.
%
\begin{verbatim}
pipe : [Send X]Unit -> [Receive X]Y -> [Abort]Y
pipe _            y             = y
pipe unit         _             = abort!
pipe (send x ? s) (receive ? r) =
  pipe (s unit) (r x)
\end{verbatim}
%
The type signature conveys several different things. The \verb!pipe!
  handler must handle all \verb!Send X! commands in its first argument
  and all \verb!Receive X! commands in its second argument. The first
  argument returns values of type \verb!Unit! and the second argument
  returns values of type \verb!Y!. The handler itself is allowed to
  perform \verb!Abort!  commands and returns a final value of type
  \verb!Y!.

Here are some things to send.
\begin{verbatim}
hello : List Char
hello =
  cons 'h' (cons 'e' (cons 'l'
    (cons 'l' (cons 'o' nil))))

space : List Char
space = cons ' ' nil

world : List Char
world =
  cons 'w' (cons 'o' (cons 'r'
    (cons 'l' (cons 'd' nil))))
\end{verbatim}
%
Here is a computation which sends them.
%
\begin{verbatim}
sender : [Send (List Char)]Unit
sender =
  sends (cons (hello
           (cons space (cons world nil))))
\end{verbatim}

Here is a \verb!main! function, which plugs \verb!sender! and
\verb!catter!  together and sends their output to the console.
%
\begin{verbatim}
main : [Abort, Console](List Unit)
main = map ouch (pipe sender catter)
\end{verbatim}
where the \verb!Console! operations are handled specially at the
top-level according to the following signature.
\begin{verbatim}
sig Console
  = inch : Char(Unit)
  | ouch : Unit(Char)
\end{verbatim}

The type system does two separate things:
\begin{itemize}
\item It ensures that value types coincide.
\item It ensures that effects required are included in effects
  enabled.
\end{itemize}

The fun of Frank is that one can say what it is to \emph{be} a
computation without saying what it is to \emph{do} it. Doing and being
are separately negotiable, and readily interleaved in different
ways. Or as Frank Sinatra put it,
\begin{center}
do be do be do
\end{center}

%% TODO: some illustration of forwarding?

%% \subsection{Modular rollback}

%% TODO: probably scrap this section

%% \subsection{Effect polymorphism with an invisible effect variable}

%% TODO: move this discussion after the type system for Frank has been
%% introduced

%% Let us consider the canonical example of a polymorphic higher-order
%% function \verb|map|.
%% \begin{verbatim}
%% data List X = nil | cons X (List X)

%% map : (X -> Y) -> List X -> List Y
%% map f nil         = nil
%% map f (cons x xs) = cons (f x) (map f xs)
%% \end{verbatim}
%% The type we have given to map is the standard one a functional
%% programmer might expect to write in a language without support for
%% effect typing.
%% %
%% Though it looks polymorphic



%% %
%% We automatically translate it into a valid Frank type by applying a
%% series of desugaring rules.

%% The valid Frank type we are after is:
%% \begin{verbatim}
%% forall e X Y.{[0]{[0]X -> [e]Y} -> [0](List X) -> [e](List Y)}
%% \end{verbatim}

%% First, for readability we allow the signature for recursively defined
%% functions to be defined on a separate line, as in Haskell.
%% %
%% Second, we allow the quantifiers to be omitted, just like in Haskell.
%% \begin{verbatim}
%% {[0]{[0]X -> [e]Y} -> [0](List X) -> [e](List Y)}
%% \end{verbatim}
%% Third, we allow the outer braces to be omitted, as the type must always be a thunk.
%% \begin{verbatim}
%% [0]{[0]X -> [e]Y} -> [0](List X) -> [e](List Y)
%% \end{verbatim}

%% %% S~\many{V} == 0, S~\many{V}
%% %% S~\many{V} == e, S~\many{V}

%% Let us distinguish between returners $\effbox{\sigs}V$ in argument
%% position, whose effects $\sigs$ must be closed, and those in tail
%% position, whose effects need not be closed. We call the former
%% \emph{ports}, and the latter \emph{pegs}.

%% Now we take advantage of some more general desugaring rules.
%% \begin{equations}
%%                       V &\equiv& \effbox{\varepsilon}V \\
%%                 V \to C &\equiv& \effbox{\emptyset}V \to C \\
%% \effbox{\sigs}(R \to C) &\equiv& \effbox{\sigs}\thunk{R \to C} \\
%% \end{equations}
%% where the first rule is restricted to pegs (it would be nonsensical
%% for ports anyway, as their effects must always be closed).

%% The first rule is the critical one. It allows us to avoid writing down
%% the effect variable. The second rule makes sense as ports being closed
%% means that writing down $\emptyset$ is redundant. The third rule is
%% superficial.

%% Applying each rule in turn we obtain:
%% \begin{verbatim}
%% [0]{[0]X -> [e]Y} -> [0](List X) -> [e](List Y)
%% [0]([0]X -> [e]Y) -> [0](List X) -> [e](List Y)
%% [0]([0]X ->    Y) -> [0](List X) ->     List Y 
%%    (   X ->    Y) ->     List X  ->     List Y 
%% \end{verbatim}

%% In order to account for non-empty effect sets, we also apply the
%% following rules:
%% \begin{equations}
%% S~\many{V} &\equiv& \varepsilon, S~\many{V} \\
%% S~\many{V} &\equiv& \emptyset, S~\many{V} \\
%% \end{equations}
%% where the first only applies to peg effect sets, and the second only
%% applies to port effect sets.

%% For instance:
%% \[
%% \bstack
%% \thunk{\effbox{\var{Abort}}\var{Int} \to \effbox{\var{State}~\var{Int}}\var{Bool}} \to \var{Int}
%% \equiv \\
%%   \qquad \thunk{\effbox{\emptyset, \var{Abort}}\var{Int} \to
%%        \effbox{\varepsilon, \var{State}~\var{Int}}\var{Bool}} \to \effbox{\varepsilon}\var{Int}
%% \estack
%% \]

%% With the syntactic sugar in place, we can now avoid writing the effect
%% variable $\varepsilon$ ever. In addition, we need never write
%% $\emptyset$ in port effect sets. It is sometimes necessary to
%% explicitly write $\emptyset$ in peg effect sets. In particular, a pure
%% top-level program returning values of type $V$ has type
%% $\effbox{\emptyset}V$.
