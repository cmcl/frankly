\section{Explicit Control Flow and Polymorphism}
\label{sec:feff}

In order to make control flow explicit, and to ease the definition of
an operational semantics, we translate Core Frank into a
call-by-push-value calculus \feff based on Kammar et al's
\lameff~\cite{KammarLO13}.

At the same time, we make polymorphism explicit. The syntax of \feff
is given in Figure~\ref{fig:feff-syntax}. Broadly, \feff types are
similar to Frank types, with an explicit division between value type
and computation types. A superficial difference is that rather than
annotating value returning computations with effects, we shift such
labels to the thunk containing the computation. The former design
seems more convenient to program with, which is why we adopt it in the
source language. The latter design leads to a more uniform
presentation of the typing rules, and matches the design of
\lameff. Another minor difference is that in \feff commands must
always be fully applied, which leads to a cleaner operational
semantics.
%
Note that $\key{let}$ is actually redundant in \feff, just as it is in
Frank and Core Frank. We include $\key{let}$ as a special construct
because doing so makes the semantics cleaner.
%
\begin{figure}
Types
\begin{syntax}
\slab{values}       &U, V   &::=& D~\many{U} \mid \forall \many{Z}.\effbox{\sigs}\thunk{C} \mid X \\ % \mid \forall X.V \\
\slab{computations} &C      &::=& \rt{V} \mid V \to C  % \mid \forall X.C \\
\\[1ex]
\slab{quantifiers}  &Z      &::=& X \mid \varepsilon \\
\slab{arguments}    &T      &::=& V \mid \sigs
\\[1ex]
\slab{polytypes}    &P      &::=& \forall \many{Z}.Q \\
\slab{monothunks}   &Q      &::=& \effbox{\sigs}\thunk{C}
\\[1ex]
\slab{signatures}   &\sig   &::=& \cdot \mid c : U(\many{V}), \sig \\
\slab{effects}      &\sigs  &::=&
  \emptyset \mid \sigs, \sig~\many{V} \mid \varepsilon
\\[1ex]
\slab{type environments}     &\Gamma &::=& \cdot \mid \Gamma, x:V \\
\end{syntax}

%% Effect extension
%% \begin{equations}
%% \sigs \oplus \emptyset               &=& \emptyset \\
%% \sigs \oplus (\sigs', \sig~\many{V}) &=& (\sigs \oplus \sigs'), \sig~\many{V} \\
%% \end{equations}

Terms
\begin{syntax}
\slab{values}       &u, v &::=& x \mid k~\many{v} \mid \Lambda \many{Z}.\thunk{d} \\
%%             \mid \Lambda Z.v \mid u~T \\
\slab{computations} &d, e &::=& \key{case}~u~\key{of}~
                                   (k~\many{x_k} \mapsto e_k)_k
                                \mid \force{(u~\many{T})} \\
                    &     &\mid& \lambda x.e \mid d~v 
                           \mid \key{ret}~v \mid \key{let}~x=e~\key{in}~e'
%%                           \mid c \\
                           \mid c~\many{v} \\
%%   \mid \Lambda Z.e \mid d~T \\
                         &&\mid& \key{handle} ~d~ \key{with}~
                                   (\handle{c~\many{x_c}}{g} \mapsto e_c)_c \medvert
                                   x        \mapsto e \\
                         &&\mid& \key{letrec}~\many{f : P = e}~\key{in}~e' \\
\end{syntax}
\caption{\feff Syntax}
\label{fig:feff-syntax}
\end{figure}



\begin{figure*}
$\boxed{\valg{v}{V}}$
\begin{mathpar}
\inferrule
  {x:V \in \Gamma}
  {\valg{x}{V}}

\inferrule
  {\valg{V_i}{v_i} \\
   k~\many{V} \in D~\many{U}}
  {\valg{D~\many{U}}{k~\many{v}}}

\inferrule
  {\compgs{e}{C}}
  {\valg{\Lambda \many{Z}.\thunk{e}}{\forall \many{Z}.\effbox{\sigs}\thunk{C}}}
%% \inferrule
%%   {\valg{v}{V} \\
%%    Z \notin \FV(\Gamma)}
%%   {\valg{\Lambda Z.v}{\forall Z.V}}
%% \inferrule
%%   {\valg{u}{\forall Z.V}}
%%   {\valg{u~T}{V[T/Z]}}
\end{mathpar}

$\boxed{\compgs{e}{C}}$
\begin{mathpar}
\inferrule
  {\valg{u}{D~\many{U}} \\
   (\comp{\Gamma, \many{x_k}:\many{V}}{\sigs}{e_k}{C})_{k~\many{V} \in D~\many{U}}}
  {\compgs{\key{case}~ u ~\key{of}~
               (k~\many{x_k} \mapsto e_k)_k}
          {C}}

\inferrule
  {\valg{u}{\forall \many{Z}.\effbox{\sigs}\thunk{C}}}
  {\compgs{\force{(u~\many{T})}}{C[\many{T}/\many{Z}]}}

\inferrule
  {\comp{\Gamma, x:V}{\sigs}{e}{C}}
  {\compgs{\lambda x.e}{V \to C}}

\inferrule
  {\compgs{d}{V \to C} \\
   \valg{v}{V}}
  {\compgs{d~v}{C}}

\inferrule
  {\valg{v}{V}}
  {\compgs{\key{ret}~v}{\rt{V}}}

\inferrule
  {\compgs{d}{\rt{V}} \\
   \comp{\Gamma, x:V}{\sigs}{e}{C}}
  {\compgs{\key{let}~x=d~\key{in}~e}{C}}

%% \inferrule
%%   {c : U(\many{V}) \in \sigs}
%%   {\compgs{c}{\many{V} \to U}}
%%
\inferrule
  {c : U(\many{V}) \in \sigs \\
   \valg{v_i}{V_i}}
  {\compgs{c~\many{v}}{\rt{U}}}

%% \inferrule
%%   {\compgs{e}{C} \\
%%    Z \notin (\FV(\Gamma) \cup \FV(\sigs))}
%%   {\compgs{\Lambda Z.e}{\forall Z.C}}
%%
%% \inferrule
%%   {\compgs{d}{\forall Z.C}}
%%   {\compgs{d~T}{V[T/Z]}}
%%
\inferrule
  {\comp{\Gamma}{\sigs \oplus \sigs'}{d}{\rt{V}} \\
   (\comp{\Gamma, \many{x_c}:\many{V},
                  g:\effbox{\sigs \oplus \sigs'}\thunk{U \to \rt{V}}}
       {\sigs}{e_c}{C})_{c : U(\many{V}) \in \sigs'} \\
   \comp{\Gamma, x:V}{\sigs}{e}{C}}
  {\compgs
     {\key{handle}~ d ~\key{with}~
           (\handle{c~\many{x_c}}{g} \mapsto e_c)_c \medvert
           x \mapsto e}
     {C}}

\inferrule
  {(\comp{\Gamma, \many{f : \forall\many{Z}.Q}}{\sigs}{e_i}{Q_i})_i \\
   \comp{\Gamma, \many{f : \forall \many{Z}.Q}}{\sigs}{d}{C}}
  {\compgs{\key{letrec}~\many{f : \forall \many{Z}.Q = e}~\key{in}~d}{C}}
\end{mathpar}

\caption{\feff Typing rules}
\label{fig:feff-typing}
\end{figure*}
%
The typing rules for \feff are given in
Figure~\ref{fig:feff-typing}. Typing in \feff is unidirectional.

Call-by-push-value calculi such as \lameff and \feff make a strict
separation between values and computations, not dissimilar from CPS or
A-normal form representations, in which all reduction takes place at
the level of computations. In such a setting it would seem most
natural to add type abstractions to computations rather than
values. However, this does not give us what we need in the presence of
effects, as we need to be able to quantify over effects. Our solution
is to build the universal quantifier into the thunk type (the
introduction rule for thunks is the place where ambient effects are
reified in a type) and build type application into forcing.

The type translation from Core Frank to \feff is given by the
homomorphic extension of the following equations:
\begin{equations}
\sem{\thunk{C}} &=& \effbox{\sem{\sigs}}\thunk{\sem{C}}, \quad \text{where }\effs{C}{\sigs} \\
\sem{\effbox{\sigs}V \to C} &=& \effbox{\sem{\sigs}}\sem{V} \to \sem{C} \\
\sem{\effbox{\sigs}V} &=& \rt{\sem{V}}\\
\end{equations}%
%
  %% drop ([Sig]{C}) = {drop(C) drop(Sig)}
  %%
  %% drop ( <V> ) Sig          = [ drop (Sig) ] drop(V)
  %% drop ( [Sig']V -> C ) Sig = [ drop (Sig') ] drop(V) -> drop(C) drop(Sig)
%
In order to simplify the definition of the term translation, as a
pre-processing step we annotate instances of polymorphic variables $f$
with the type arguments they are instantiated with and commands with
their arities.
%
The term translation is a call by value embedding of Core Frank into
\feff:
%
\begin{equations}
\sem{x} &=& \key{ret}~x \\
\sem{f^{\many{T}}} &=& \key{ret}~\thunk{f~\sem{\many{T}}} \\
\sem{c^n} &=& \key{ret}~\thunk{\lambda x_1 \dots x_n.c~x_1~\dots~x_n} \\
\\
\sem{k~\many{v}} &=& \key{let}~\many{x}=\sem{\many{v}} ~\key{in}~\key{ret}~(k~\many{x}) \\
\sem{\thunk{e}} &=& \key{ret}~\thunk{\sem{e}} \\
\\
%% \sem{c} &=& c \\
%% \sem{c~\many{v}} &=& \key{let}~\many{x}=\sem{\many{v}} ~\key{in}~c~\many{x} \\
\sem{d~v} &=& \key{let}~x=\sem{v} ~\key{in}~\sem{d}~x \\
\sem{\force{u}} &=& \key{let}~x=\sem{u} ~\key{in}~\force{x} \\
\sem{\key{letrec}~\many{f : \forall \many{Z}.Q = e}~\key{in}~d} &=&
  \key{letrec}~\many{f : \sem{\forall \many{Z}.Q} = \sem{e}}~\key{in}~\sem{d} \\
\\
\sem{\lambda x.e} &=& \lambda x.\sem{e} \\
\multicolumn{3}{@{}l}{\sem{\key{case}~u~\key{of}~(k~\many{x_k} \mapsto e_k)_k} =} \\
\multicolumn{3}{@{}l}
  {\qquad \key{let}~x=\sem{u} ~\key{in}~
    \key{case}~x~\key{of}~(k~\many{x_k} \mapsto \sem{e_k})_k} \\
\multicolumn{3}{@{}l}{\sem{\key{handle}~d~\key{with}~(\handle{c~\many{x_c}}{g_c} \mapsto e_c)_c \medvert x \mapsto e} =} \\
\multicolumn{3}{@{}l}
  {\qquad \key{handle}~\sem{d}~\key{with}~ 
            (\handle{c~\many{x_c}}{g_c} \mapsto \sem{e_c})_c
            \medvert x \mapsto \sem{e}} \\
\end{equations}%
%% \[
%% \bl
%% \seml
%%   \inferrule
%%     {f:\forall \varepsilon \many{X}.V \in \Gamma \\
%%      \dom(\theta) = \set{\varepsilon, \many{X}} \\
%%      \theta(\varepsilon) = \sigs}
%%     {\makesgs{f}{\theta(V)}}
%% \semr \qquad \\[3ex]
%% \hfill = \force{(f~\sigs~\many{\theta(X)})} \\
%% \el
%% \]
%

\begin{proposition}
The translation $\sem{-}$ from Core Frank to \feff is type preserving.
\begin{itemize}
\item If $\makesgs{u}{V}$ then $\comp{\sem{\Gamma}}{\sem{\sigs}}{\sem{u}}{\rt{\sem{V}}}$.
\item If $\hasgs{V}{v}$   then $\comp{\sem{\Gamma}}{\sem{\sigs}}{\sem{v}}{\rt{\sem{V}}}$.
\item If $\cangs{d}{C}$   then $\comp{\sem{\Gamma}}{\sem{\sigs}}{\sem{d}}{\sem{C}}$.
\item If $\doesg{C}{e}$ and $\effs{C}{\sigs}$ then
  $\comp{\sem{\Gamma}}{\sem{\sigs}}{\sem{e}}{\sem{C}}$.
\end{itemize}
\end{proposition}

Kammar et al~\cite{KammarLO13} classify a number of different
varieties of handler.
%
The handlers in \feff (and indeed Frank) are \emph{shallow} in that,
unlike Plotkin and Pretnar's original deep handlers, the handler is
not automatically rewrapped around the continuation when handling a
command. Roughly, deep handlers perform a fold over computations,
whereas shallow handles perform a case split. Deep handlers are
denotationally better behaved than shallow handlers, but shallow
handlers sometimes appear more convenient to program with.
%
The handlers in \feff (and indeed Frank) are polymorphic forwarding
handlers. The form of polymorphism amounts to a kind of row
polymorphism with shadowing, unlike the more conventional kind of row
polymorphism suggested by Kammar et al.

%% \[
%% \bl
%% \seml
%%   \inferrule
%%     {\dom(\theta) = \set{\varepsilon} \cup \FV(V) \backslash (\FV(\Gamma) \cup \FV(\sigs)) \\
%%      \theta(\varepsilon) = \sigs}
%%     {\val{\Gamma, x:V}{x}{\theta(V)}}
%% \semr
%% = \\
%%   \inferrule
%%     {\dom(\theta) = \set{\varepsilon} \cup \FV(V) \backslash (\FV(\Gamma) \cup \FV(\sigs)) \\
%%      \theta(\varepsilon) = \sigs}
%%     {\val{\Gamma, x:\forall \many{Z}.V}{x~\theta(\many{Z})}{\theta{V}}} \\
%% \el
%% \]

%%  LocalWords:  Lindley Conor Strathclyde Plotkin Pretnar's Multi et
%%  LocalWords:  effectful modularity polymorphism Kammar al Pretnar
%%  LocalWords:  Hindley Milner Oury SML Haskell monadic Kiselyov DSL
%%  LocalWords:  Idris Brady's equational multi unary al's datatypes
%%  LocalWords:  datatype Vn typecheck Ri catter polytypes inferable
%%  LocalWords:  checkable letrec monomorphic reifying desugars ret
%%  LocalWords:  homomorphic monothunks reified pre rewrapped monads
%%  LocalWords:  denotationally Swierstra Visscher Filinski Daan Koka
%%  LocalWords:  Filinski's Leijen's Swamy monad

