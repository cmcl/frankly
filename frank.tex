\section{Type System}
\label{sec:frank}

In this section we give a formal presentation of Frank's type system.
%
\begin{figure}
Types
\begin{syntax}
\slab{values}       &U, V          &::=& D~\many{U} \mid \thunk{C} \mid X \\
\slab{computations} &C             &::=& R \mid R \to C \\
\slab{returners}    &R             &::=& \effbox{\sigs}V
\\[1ex]
\slab{quantifiers}  &Z             &::=& X \mid \varepsilon \\
\slab{polytypes}    &P             &::=& \forall \many{Z}.Q \\
\slab{thunks}       &Q             &::=& \thunk{C}
\\[1ex]
\slab{signatures}   &\sig~\many{X} &::=& \cdot \mid c : U(\many{V}), \sig~\many{X} \\
\slab{effects}      &\sigs  &::=&
  \emptyset \mid \sigs, \sig~\many{V} \mid \varepsilon
\\[1ex]
\slab{type environments}     &\Gamma        &::=& \cdot \mid \Gamma, x:V \mid f:P \\
\end{syntax}

Terms
\begin{syntax}
\slab{inferable values}&u       &::=& x \mid f \mid c \mid d \\
\slab{checkable values}&v       &::=& u \mid k~\many{v} \mid \thunk{e}
\\[1ex]
\slab{inferable computations}&d &::=& \force{u} \mid d~e \\
%                               \mid  \key{let}~x=d~\key{in}~d' \\
                             &&\mid& \key{letrec}~\many{f : P = e}~\key{in}~d \\
\slab{checkable computations} &e &::=& v \mid \many{r} \mapsto e \mid () \mid e \medvert e
\\[1ex]
\slab{value patterns}&p       &::=& x \mid k~\many{p}                              \\
\slab{computation patterns}&r &::=& p \mid \handle{c~\many{p}\,}{g} \mid \force{x} \\
\end{syntax}

\caption{Frank Syntax}
\label{fig:frank-syntax}
\end{figure}
%
The syntax of Frank types and terms is given in
Figure~\ref{fig:frank-syntax}. The types are divided into value types
and computation types in a similar fashion to Levy's
call-by-push-value calculus~\cite{Levy2004}.
%
Value types are datatypes ($D~\many{U}$), suspended $C$ computations
($\thunk{C}$), otherwise known as \emph{thunks}, or type variables
($X$).

Computation types are constructed from \emph{returners}.  A returner
$(\effbox{\sigs}V)$ represents the type of a computation that returns
values of type $V$ while performing effects in $\sigs$.
%
In general, computation types represent multi-handlers. The type
$\effbox{\sigs_1}V_1 \to \dots \to \effbox{\sigs_n}V_n \to
\effbox{\sigs}V$,
%
is the type of an $n$-handler. For each argument type
$\effbox{\sigs_i}V_i$ the multi-handler must handle effects in
$\sigs_i$ on that argument. Such an $n$-handler handles all of its
arguments simultaneously. As a result of handling its arguments it
returns a value of type $V$ and may perform effects in $\sigs$.
%
We often write $\many{\effbox{\sigs}V} \to C$ as an abbreviation for
$\effbox{\sigs_1}V_1 \to \dots \to \effbox{\sigs_n}V_n \to C$.

Polytypes are restricted to thunks. Effect polymorphism is restricted
to a single effect variable $\varepsilon$, which in practice Frank
programmers need never write.

Datatypes and effect signatures are declared at the top-level. An
effect signature $S~\many{X}$ consists of a collection of command
declarations of the form $c:U(\many{V})$, denoting that command $c$
takes arguments of types $\many{V}$ and returns a value of type
$U$. The types $\many{V}$ and $U$ may all depend on $\many{X}$. Each
command many appear only once in a signature, and each command may
appear in only one signature.

An effect set is a sequence of signatures initiated either with the
empty effect $\emptyset$ (yielding a \emph{closed effect set}) or the
only effect variable $\varepsilon$ (yielding an \emph{open effect
  set}). Order is important, as repeats are permitted, in which case
the right-most signature overrides all others with the same name.

Shadowing arises naturally in two ways. First, given an open effect
set $\sigs$ we may substitute an arbitrary effect set for
$\varepsilon$. Second, we define a notion of effect extension: $\sigs
\oplus \sigs'$ is the \emph{extension} of effect set $\sigs$ with
closed effect set $\sigs'$, formally:
%
\[\ba{@{\sigs~\oplus~}l@{~}l@{}}
\emptyset               &= \sigs \\
(\sigs', \sig~\many{V}) &= (\sigs \oplus \sigs'), \sig~\many{V} \\
\ea\]

Type environments distinguish monomorphic and polymorphic variables.

Just as with the types, Frank terms are separated into value terms and
computation terms. With future extensions, such as dependent types, in
mind, Frank adopts a bidirectional typing
discipline~\cite{PierceT00}. Thus terms are further sub-divided into
those whose type is inferable, and those that may be checked against a
type.

Frank is less strict about the separation between value and
computation terms than call-by-push-value is. For instance, inferable
computations can sometimes be treated as inferable values. This is a
deliberate design decision, with the aim of making Frank convenient to
program with.

%%%%% Frank

\begin{figure*}[float]
$\boxed{\makes{\Gamma}{\sigs}{u}{V}}$
\begin{mathpar}
\inferrule
  {x:V \in \Gamma}
  {\makesgs{x}{V}}

%% \inferrule
%%   {\dom(\theta) = \set{\varepsilon} \cup \FV(V) \backslash (\FV(\Gamma) \cup \FV(\sigs)) \\
%%    \theta(\varepsilon) = \sigs}
%%   {\makes{\Gamma, x:V}{\sigs}{x}{\theta(V)}}

\inferrule
  {f:\forall \many{Z}.Q \in \Gamma \\
   \dom(\theta) = \set{\many{Z}} \\
   \varepsilon \in \dom(\theta) \implies \theta(\varepsilon) = \sigs}
  {\makesgs{f}{\theta(Q)}}

\inferrule
  {c : U(\many{V}) \in \sigs}
  {\makesgs{c}{\thunk{\many{\effbox{}V} \to \effbox{\sigs}U}}}

\inferrule
  {\cangs{d}{\effbox{\sigs}V}}
  {\makesgs{d}{V}}
\end{mathpar}

$\boxed{\has{\Gamma}{\sigs}{V}{v}}$

\begin{mathpar}
\inferrule
  {\makesgs{u}{U} \\ U = V}
  {\hasgs{V}{u}}

\inferrule
  {(\hasgs{V_i}{v_i})_i \\
   k~\many{V} \in D~\many{U}}
  {\hasgs{D~\many{U}}{k~\many{v}}}

\inferrule
  {\does{\Gamma}{C}{e} \\ e \text{ covers } C}
  {\hasgs{\thunk{C}}{\thunk{e}}}
\end{mathpar}

$\boxed{\can{\Gamma}{\sigs}{d}{C}}$

\begin{mathpar}
\inferrule
  {\makesgs{u}{\thunk{C}} \\ \effs{C}{\sigs}}
  {\cangs{\force{u}}{C}}

\inferrule
  {\cangs{d}{\effbox{\sigs'}V \to C} \\
   \doesg{\effbox{\sigs \oplus \sigs'}V}{e}}
  {\can{\Gamma}{\sigs}{d~e}{C}}

%% \inferrule
%%   {c : U(\many{V}) \in \sigs \\
%%    (\hasgs{V_i}{v_i})_i}
%%   {\cangs{c~\many{v}}{\effbox{\sigs}U}}

%% \inferrule
%%   {\cangs{d}{V} \\
%%    \can{\Gamma, x:V}{\sigs}{d'}{C}}
%%   {\cangs{\key{let}~x=d~\key{in}~d'}{C}}

\inferrule
  {(\does{\Gamma, \many{f : \forall \many{Z}.Q}}{e_i}{Q_i})_i \\
   \can{\Gamma, \many{f : \forall \many{Z}.Q}}{\sigs}{d}{C}}
  {\cangs{\key{letrec}~\many{f : \forall \many{Z}.Q = e}~\key{in}~d}{C}}
\end{mathpar}

$\boxed{\doesg{C}{e}}$

\begin{mathpar}
\inferrule
  {\hasgs{V}{v}}
  {\doesg{\effbox{\sigs}V}{v}}

\inferrule
  {\many{\sigs'} \text{ closed} \\
   (\pat{\effbox{\sigs'_i}V_i}{r_i}{\Delta_i})_i \\
   \does{\Gamma, \many{\Delta}}{C}{e}}
  {\doesg{\many{\effbox{\sigs'}V} \to C}{\many{r} \mapsto e}}

\inferrule
  {V\, \text{uninhabited}}
  {\doesg{\effbox{\sigs'}V \to C}{()}}

\inferrule
  {\doesg{R \to C}{e} \\ \doesg{R \to C}{e'}}
  {\doesg{R \to C}{e \medvert e'}}
\end{mathpar}

$\boxed{\effs{C}{\sigs}}$

\begin{mathpar}
\inferrule
  { }
  {\effs{\effbox{\sigs}V}{\sigs}}

\inferrule
  {\effs{C}{\sigs}}
  {\effs{R \to C}{\sigs}}
\end{mathpar}

$\boxed{\pat{V}{p}{\Delta}}$

\begin{mathpar}
\inferrule
  { }
  {\pat{V}{x}{x:V}}

\inferrule
  {k~\many{V} \in D~\many{U} \\
   (\pat{V_i}{p_i}{\Delta_i})_i}
  {\pat{D~\many{U}}{k~\many{p}}{\many{\Delta}}}
\end{mathpar}

$\boxed{\pat{R}{r}{\Delta}}$

\begin{mathpar}
\inferrule
  {\pat{V}{p}{\Delta}}
  {\pat{\effbox{\sigs}V}{p}{\Delta}}

\inferrule
  {c:U(\many{V}) \in \sigs \\
   (\pat{V_i}{p_i}{\Delta_i})_i}
  {\pat{\effbox{\sigs}V}
       {\handle{c~\many{p}}{g}}
       {\many{\Delta}, g:\thunk{\effbox{\emptyset}U \to \effbox{\sigs}V}}}

\inferrule
  { }
  {\pat{\effbox{\sigs}V}{\force{x}}{x:\thunk{\effbox{\sigs}V}}}
\end{mathpar}

\caption{Frank Typing Rules}
\label{fig:frank-typing}
\end{figure*}

The typing rules for Frank are given in Figure~\ref{fig:frank-typing}.
%
The judgement $\makesgs{u}{V}$ says that given type environment
$\Gamma$, ambient effects $\sigs$, and inferable value term $u$, then
we can \emph{infer} that the type of $u$ is $V$.
%
The judgement $\hasgs{V}{v}$ says that given type environment
$\Gamma$, ambient effects $\sigs$, value type $V$, and checkable value
term $v$, then we can \emph{check} that $V$ is the type of $v$.
%
The judgement $\cangs{d}{C}$ says that given type environment
$\Gamma$, ambient effects $\sigs$, and inferable computation term $d$,
then we can \emph{infer} that the type of $d$ is $C$.
%
The judgement $\doesg{C}{e}$ says that given type environment
$\Gamma$, computation type $C$, and checkable computation term $e$,
then we can \emph{check} that $C$ is the type of $e$.

The types of monomorphic variables ($x$) are simply looked up in the
type environment. The types of polymorphic variables ($f$) are looked
up and instantiated. In practice this means applying a simple
unification-based algorithm.
%
The type of a command ($c$) is looked up from the ambient effects.
%
An inferable computation ($d$) is also an inferable value, providing
it is a returner whose effects agree with the ambient effects.

Any inferable value ($u$) is also checkable against its inferred
type. Datatype ($k~\many{v}$) and thunk ($\thunk{e}$) terms are
checkable by checking their components. The side condition on the
thunk introduction rule requires that the pattern matching clauses of
$e : C$ cover $C$.

A thunk $u$ can by forced ($u!$) if its inferred type agrees with the
ambient effects. To infer the type of a handler application $d~e$, we
first infer the type of $d$, and then check that the argument matches
the inferred argument type. Note that the effect set on an argument
must be closed, and thus we may extend the ambient effect set with the
argument effect set when checking the argument.
%
%% The term $\key{let}~x=d~\key{in}~d'$ binds the result of running $d$
%% in $d'$.
%
The term $\key{letrec}~\many{f : P = e}~\key{in}~d$ binds the mutually
recursive polymorphic functions $\many{f:P}$ in $d$.

Any term $v$ that type checks with ambient effects $\sigs$ at value
type $V$ also type checks at computation type $\effbox{\sigs}V$.
%
A multi-handler of type $\many{\effbox{\sigs'}V} \to C$ is built by
composing clauses of the form $\many{r} \mapsto e$, where $\many{r}$
is a sequence of \emph{computation patterns} whose variables are bound
in $e$.
%
The $e \medvert e'$ construct composes the clauses of $e$ with those
of $e'$. A multi-handler is defined by a collection of clauses that
provides complete coverage of the input types.
%
The $()$ construct allows empty sets of clauses to be constructed in
the event that the value type of a function argument is uninhabited.

We preclude composition of non-clauses (i.e. returners) by
constraining $e \medvert e'$ to have function type.
%
It is perfectly legitimate to compose clauses with different numbers
of arguments, though as a preliminary part of pattern matching
compilation, we first transform the program to ensure that all
composed clauses do have the same number of arguments.

Value patterns are standard, consisting of variable patterns ($x$) and
datatype constructor patterns ($k~\many{p}$).

Computation patterns are more interesting. Any value pattern $p$ is
also a computation pattern (used to match against the value returned
by the returner computation).
%
A request pattern $\handle{c~\many{p}}{g}$ matches against a
computation of the form $\force{c}~\many{v}$ if the values $v$ match
against $p$. Furthermore, it also binds $g$ to the continuation of the
computation delimited by the nearest enclosing multi-handler. This is
where the real power of multi-handlers arises. A thunk pattern
$\force{x}$ matches any computation reifying it as a thunk bound to
$x$.

%% let x = e in e' == bind! e {x -> e'}
%% bind e e' == let x=e in e' x

\paragraph{Sequencing Computations}
We write $\key{let}~x=e~\key{in}~e'$ as syntactic sugar for
$\var{bind}~e~\thunk{x \mapsto e'}$, where $\var{bind}$ is defined in
Section~\ref{sec:examples}. More verbosely:
%
\[
\bstack
\key{let}~x=e~\key{in}~e' \equiv \\
\quad \key{letrec}~
  (\var{bind} :
    \forall \varepsilon~X~Y.
       X \to \thunk{X \to \effbox{\varepsilon}Y} \to \effbox{\varepsilon}Y) = \\
  \qquad x~f~\mapsto~f~x~\key{in}~\force{\var{bind}}~e~\thunk{x \mapsto e}
\estack
\]

\subsection{Effect Polymorphism with an Invisible Effect Variable}

We now give a brief description of how syntactic sugar allows Frank
programmers to omit effect variables completely.
%
Consider the type of \verb!map! in Section~\ref{sec:examples}:
\[
\thunk{X \to Y} \to \var{List~X} \to \var{List~Y}
\]
Apart from perhaps the curly braces, this looks pretty much the same
as the type a functional programmer might expect to write in a
language without support for effect typing.

In fact, this type desugars into the rather more verbose:
\[
\thunk{\effbox{\emptyset}X \to \effbox{\varepsilon}Y}
  \to \effbox{\emptyset}(\var{List~X}) \to \effbox{\varepsilon}(\var{List~Y})
\]

Let us distinguish between returners $\effbox{\sigs}V$ in argument
position, whose effects $\sigs$ must be closed, and those in tail
position, whose effects need not be closed. We call the former
\emph{ports}, and the latter \emph{pegs}.

Observe that ports are closed, so it is never necessary to write the
$\emptyset$ in their effects. Pegs, on the other hand, may be open or
closed. We adopt the convention that $\varepsilon$ may be omitted from
the start of a peg's effects. Thus, if we know it is a peg, then $V$
means $\effbox{}V$, which means $\effbox{\varepsilon}V$, and
$\effbox{\var{Abort}}V$ means $\effbox{\varepsilon, \var{Abort}}V$.

We now summarise the syntactic sugar. For ports and pegs:
\begin{equations}
V &\equiv& \effbox{}V \\
\end{equations}
For ports:
\begin{equations}
\effbox{\many{S~\many{V}}}U &\equiv& \effbox{\emptyset, \many{S~\many{V}}}U \\
\end{equations}
For pegs:
\begin{equations}
\effbox{\many{S~\many{V}}}U &\equiv& \effbox{\varepsilon, \many{S~\many{V}}}U \\
\end{equations}

With this syntactic sugar in place, we can now avoid writing the
effect variable $\varepsilon$ in Frank programs, ever. In addition, we
need never write $\emptyset$ in port effect sets. It is sometimes
necessary to explicitly write $\emptyset$ in peg effect sets. In
particular, a pure top-level program returning values of type $V$ has
type $\effbox{\emptyset}V$.

%% [already mentioned in the example section]
%%
%% Another useful piece of syntactic sugar is for function
%% application. We often write $u~e$ for $\force{u}~e$.

%% TODO: get rid of GIBBERISH




%% As is made apparent by the call-by-value translation in
%% Section~\ref{sec:feff}, Frank is morally a call-by-value language
%% (though multi-handlers complicate the picture somewhat).




%% Frank has a number of distinctive features.

%% Inspired by Levy's call-by-push-value (CBPV) calculus, Frank makes an
%% explicit distinction between computation and value types. However,
%% where CBPV also has a clear distinction between value terms and
%% computation terms, Frank is less rigid. The motivation for this design
%% is to reduce boilerplate in source programs. Frank combines the
%% advantages of CBPV, call-by-value (a la ML), and effect typing.

%% With future extensions for dependent types in mind, Frank adopts a
%% bidirectional typing discipline. A side-effect is that one cannot
%% write beta-redexes in Frank. We do not view this as a disadvantage.

%% Inspired by the observation that applying a function to a value is
%% equivalent to applying a handler to a pure computation, handlers and
%% functions are unified in Frank. Handlers are further generalised to
%% multi-handlers, which handle multiple computations simulataneously.

%% Frank provides a novel form of effect polymorphism, in which it is
%% never necessary to mention the names of effect variables.

%% We explain the semantics of Frank programs through a series of program
%% transformations.

%% First, we take a fairly standard approach to compiling away pattern
%% matching (Core Frank). As we may match simultaneously against multiple
%% side-effecting computations, we must be careful about
%% order. Optionally, we can expose non-exhaustive/incomplete pattern
%% matching as concrete effects.

%% Second, we make the distinction between values terms and computations
%% explicit, as in CBPV (\impeff).

%% Finally, we make polymorphism explicit. Our ultimate target is \feff,
%% an effect-polymorphic variant of Kammar et al's CBPV-inspired
%% lambda-eff calculus. The small-step semantics of \feff is a
%% straightforward variant of that of lambda-eff.
