\section{Introduction}
\label{sec:intro}

% TODO: more background on algebraic effects and handlers?
% CMCL: Need to map directly (categorically?) back to Plotkin+Power (P+P)
% it is unsatisfying not to have a more explicit relationship even if the
% connection between P+P and Frank is clearer than, say, Idris and P+P

Plotkin and Power's algebraic effects~\cite{PlotkinP01a, PlotkinP01b,
  PlotkinP02, PlotkinP03}, in conjunction with Plotkin and Pretnar's
handlers for algebraic effects~\cite{PlotkinP09, PlotkinP13}, provide
a compelling foundation for effectful programming.
%
By separating effect signatures from their implementation, algebraic
effects provide a high degree of modularity, allowing programmers to
express effectful programs independently of the concrete
interpretation of their effects. A handler is an interpretation of the
effects of an algebraic computation.

%% This paper explores the design of Frank, a strict functional
%% programming language designed from the ground up around a novel
%% variant of Plotkin and Pretnar's \emph{effect
%%   handler}~\cite{PlotkinP09, PlotkinP13} abstraction.

%% Inspired by Levy's call-by-push-value calculus~\cite{Levy2004}, Frank
%% makes an explicit distinction between computation and value
%% types. Frank combines the advantages of call-by-push-value,
%% call-by-value, and effect typing.

Effect handlers in Frank generalise functions. An effect handler acts
as an interpreter for a specified set of commands whose signatures are
statically tracked by the type system. A function is the special case
of an effect handler whose command set is empty.

The contributions of this paper are:
\begin{itemize}
\item Frank, a strict functional programming language featuring a
  bidirectional effect type system, effect polymorphism, and effect
  handlers.
\item A novel approach to effect polymorphism which avoids all mention
  of effect variables, crucially relying on the observation that one
  must always instantiate the effects of a function being applied with
  the current \emph{ambient} effects.
\item The combination of pattern matching and effect handlers in such
  a way that incomplete or ambiguous pattern matching can be realised
  as concrete effects that can be handled however the programmer
  chooses using further effect handlers.
\item Multi-handlers as both an abstraction for handling multiple
  computations over different effect sets simultaneously and a
  characterisation of effect-handlers as generalised functions.
\item A description of pattern matching compilation from Frank into a
  core language, \emph{Core Frank}.
\item A translation from Core Frank into \feff, a polymorphic variant
  of the \lameff calculus of Kammar et al~\cite{KammarLO13}, which in
  turn extends Levy's call-by-push-value calculus~\cite{Levy2004}.
\item A straightforward small-step operational semantics for \feff,
  yielding, in combination with the translation to \feff, a type
  soundness result for Frank.
\end{itemize}

A number of other languages and libraries are built around effect
handlers and algebraic effects.

Bauer and Pretnar's Eff~\cite{BauerP12}. A significant difference
between Frank and the original version of Eff~\cite{BauerP12} is that
the latter provides no support for effect typing. Recently Bauer and
Pretnar have designed an effect type system for
Eff~\cite{BauerP13}. Their implementation~\cite{Pretnar13} supports
Hindley-Milner type inference, and the type system incorporates effect
sub-typing. In contrast, Frank uses bidirectional type inference, and
avoids sub-typing altogether.

Handlers in action~\cite{KammarLO13}. In previous work with Kammar and
Oury~\cite{KammarLO13}, the first author designed and experimented
with a number of effect handler libraries for languages ranging from
Racket, to SML, to Haskell. Apart from the Haskell library, these
other libraries have no effect typing support. The Haskell library
takes advantage of type classes to simulate an effect type system not
entirely dissimilar to that of Frank. As Haskell is lazy, the Haskell
library cannot be used to write direct-style effectful programs - one
must instead adopt a monadic style. Furthermore, although there are a
number of ways of almost simulating effect type systems in Haskell,
none is without its flaws. Kiselyov et al~\cite{KiselyovSS13} have
designed another Haskell library for effect handlers, making a
different collection of design choices.

Brady~\cite{Brady13} has designed a library and DSL for programming
with effects in his dependently typed Idris language. Like the Haskell
libraries, Brady's library currently requires the programmer to write
effectful code in a monadic style.

In Plotkin and Power’s setting, one defines algebraic effects with
respect to an equational theory. In all of the above implementations,
and in Frank, the equational theory is taken to be the free theory, in
which there are no equations.
%

The second author has been plotting Frank since
2007~\cite{McBride07}. He has implemented a prototype of a previous
version of Frank~\cite{McBride12}. The design described in the current
paper has much in common with that implementation, but there are some
syntactic and semantic differences. The most important change in the
current design is the introduction of multi-handlers as a
generalisation of both functions and handlers.

The rest of the paper is structured as follows.
%
Section~\ref{sec:examples} introduces Frank by
example. Section~\ref{sec:frank} presents a type system for
Frank. Section~\ref{sec:core} describes how to elaborate
multi-handlers and pattern matching into Core Frank, a language of
plain call-by-value functions, explicit case analysis and unary
handler constructs. Section~\ref{sec:feff} gives a call-by-value
embedding of Core Frank into \feff, a variant of Kammar et al's
\lameff calculus with shallow handlers, explicit polymorphism and
general recursion. Section~\ref{sec:semantics} gives a semantics for
\feff, which when composed with pattern matching and the call-by-value
embedding, yields a semantics for Frank. Section~\ref{sec:related}
outlines related work and Section~\ref{sec:future} discusses future
work.

%%  LocalWords:  Lindley Conor Strathclyde Plotkin Pretnar's Multi et
%%  LocalWords:  effectful modularity polymorphism Kammar al Pretnar
%%  LocalWords:  Hindley Milner Oury SML Haskell monadic Kiselyov DSL
%%  LocalWords:  Idris Brady's equational multi unary al's datatypes
%%  LocalWords:  datatype Vn typecheck Ri catter polytypes inferable
%%  LocalWords:  checkable letrec monomorphic reifying desugars ret
%%  LocalWords:  homomorphic monothunks reified pre rewrapped monads
%%  LocalWords:  denotationally Swierstra Visscher Filinski Daan Koka
%%  LocalWords:  Filinski's Leijen's Swamy monad
