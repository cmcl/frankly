\section{Related Work}
\label{sec:related}

We have discussed much of the related work throughout the paper. Here
we briefly mention some other related work.

A natural implementation for handlers is to use \emph{free
  monads}~\cite{KammarLO13}. Swierstra~\cite{Swierstra08} illustrates
how to write effectful programs with free monads in Haskell, taking
advantage of type-classes to provide a certain amount of modularity.

Inspired by Bauer and Pretnar's Eff, Visscher has implemented the
\texttt{effects} library~\cite{Visscher12}. The key idea is to layer
continuation monads in the style of Filinski~\cite{Filinski99}, using
Haskell type classes to automatically infer lifting between layers.

Filinski's work on monadic reflection and layered monads is closely
related to effect handlers~\cite{Filinski10}. Monadic reflection
supports a similar style of composing effects. The key difference is
that monadic reflection interprets monadic computations in terms of
other monadic computations, rather than abstracting over and
interpreting operations

Languages other than Frank that attempt to elide some effect variables
from source code include Links~\cite{LindleyC12} and Daan Leijen's
Koka~\cite{Leijen13}. Neither eliminates effect variables altogether.

Swamy et al~\cite{SwamyGLH11} add support for monads in ML, supporting
direct-style effectful programming in a strict language. Unlike Frank,
their system is based on monad transformers rather than effect
handlers.
